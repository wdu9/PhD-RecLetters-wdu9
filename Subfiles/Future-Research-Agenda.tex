% -*- mode: LaTeX; TeX-PDF-mode: t; -*- # Tell emacs file type (for syntax coloring)
% Add the listed directories to the search path
% (allows easy moving of files around later)
% these paths are searched AFTER local config kpsewhich

% *.sty, *.cls
\makeatletter
\def\input@path{{@resources/texlive/texmf-local/tex/latex/}
        ,{@resources/texlive/texmf-local/bibtex/bst/},
        ,{@resources/texlive/texmf-local/bibtex/bib/},
        ,{@local/}
        }
\makeatother
\makeatletter
\def\bibinput@path{{@resources/texlive/texmf-local/tex/latex/}
        ,{@resources/texlive/texmf-local/bibtex/bst/},
        ,{@resources/texlive/texmf-local/bibtex/bib/},
        ,{@local/}
        }
\makeatother
  % allow latex to find custom stuff
% LaTeX path to the root directory of the current project, from the directory in which this file resides
% and path to econtexPaths which defines the rest of the paths like \FigDir
\providecommand{\econtexRoot}{}\renewcommand{\econtexRoot}{.}

\documentclass[\econtexRoot/Letter]{subfiles}
\onlyinsubfile{\externaldocument{\econtexRoot/Letter}} % Get xrefs from main file; only works if main file has already been compiled

\begin{document}

% Details on future research agenda
Will’s job market paper has laid the foundation for an innovative research agenda that he is eager to advance. The first item on his agenda to delve into models of the sources of unemployment scarring and how those channels may be influenced by the business cycle.% This made it sound like nobody had thought of this before

One plausible explanation is that income uncertainty pushes households to accept jobs at lower wages as a precautionary measure. Although the idea of uncertainty in driving precautionary mismatches between workers and firms has recently been explored in \cite{HuangQiu2022}, no paper, to the best of my knowledge, has explored the extent to which income uncertainty induces households to be scarred. Will intends to pursue this idea and has developed the intense technical aptitude required to tackle it. The dynamics in unemployment  risk beliefs documented in his second chapter can support this idea and highlight that perceptions in unemployment risk can induce precautionary self-induced scarring. Furthermore, Will's job market paper reveals the importance of temporary layoffs in preventing both a sluggish recovery and a permanent rise in income inequality following the COVID Recession.

The manner in which the JMP models layoffs, however, is purposely kept simple to convey the broad message that scarring matters to macroeconomics. Will is working on a new paper in which he micro-founds scarring following \cite{Gertler2022} to explore the extent to which fiscal stimulus during the COVID Recession induced firms to engage in temporary layoffs as opposed to permanent layoffs. \cite{Gertler2022} build a search and matching model to demonstrate that the \href{https://home.treasury.gov/policy-issues/coronavirus/assistance-for-small-businesses/paycheck-protection-program}{Paycheck Protection Program} increased employment largely by preventing temporary layoffs from becoming permanent. Integrating the search and matching micro foundations that underlie this finding into Will's general equilibrium framework that captures realistic micro consumption behavior would allow for a rigorous evaluation of the Paycheck Protection Program's role in enabling a swift recovery.  This would shed light on how fiscal policy contributed to the rapid rebound from the COVID Recession through its effects on temporary layoffs.

A related topic is whether and how unemployment insurance (UI) can mitigate macro effects induced from unemployment scarring. He emphasizes that our understanding of the transmission of UI can be enhanced by considering the fact, documented by his classmate Wonsik Ko (\href{https://www.dropbox.com/scl/fi/ps41vz5vu0o4u8cvu315b/JMP_draft_wonsik_ko.pdf?rlkey=1j1ysvonnn577154u7xb0829x&e=1&dl=0}{paper}), that scarring effects are attenuated when the unemployed receive more generous UI extensions; this allows them to find jobs they are more satisfied with and can work more hours in. Expanding the framework in Will's job market paper to include job satisfaction could offer a new perspective on how UI can stabilize the long-run path of output.


\onlyinsubfile{% Allows two (optional) supplements to hard-wired \texname.bib bibfile:
% system.bib is a default bibfile that supplies anything missing elsewhere
% Add-Refs.bib is an override bibfile that supplants anything in \texfile.bib or system.bib
\provideboolean{AddRefsExists}
\provideboolean{systemExists}
\provideboolean{BothExist}
\provideboolean{NeitherExists}
\setboolean{BothExist}{true}
\setboolean{NeitherExists}{true}

\IfFileExists{\econtexRoot/Add-Refs.bib}{
  % then
  \typeout{References in Add-Refs.bib will take precedence over those elsewhere}
  \setboolean{AddRefsExists}{true}
  \setboolean{NeitherExists}{false} % Default is true
}{
  % else
  \setboolean{AddRefsExists}{false} % No added refs exist so defaults will be used
  \setboolean{BothExist}{false}     % Default is that Add-Refs and system.bib both exist
}

% Deal with case where system.bib is found by kpsewhich
\IfFileExists{/usr/local/texlive/texmf-local/bibtex/bib/system.bib}{
  % then
  \typeout{References in system.bib will be used for items not found elsewhere}
  \setboolean{systemExists}{true}
  \setboolean{NeitherExists}{false}
}{
  % else
  \typeout{Found no system database file}
  \setboolean{systemExists}{false}
  \setboolean{BothExist}{false}
}

\ifthenelse{\boolean{showPageHead}}{ %then
  \clearpairofpagestyles % No header for references pages
  }{} % No head has been set to clear

\ifthenelse{\boolean{BothExist}}{
  % then use both
  \typeout{bibliography{\econtexRoot/Add-Refs,\econtexRoot/\texname,system}}
  \bibliography{\econtexRoot/Add-Refs,\econtexRoot/\texname,system}
  % else both do not exist
}{ % maybe neither does?
  \ifthenelse{\boolean{NeitherExists}}{
    \typeout{bibliography{\texname}}
    \bibliography{\texname}}{
    % no -- at least one exists
    \ifthenelse{\boolean{AddRefsExists}}{
      \typeout{bibliography{\econtexRoot/Add-Refs,\econtexRoot/\texname}}
      \bibliography{\econtexRoot/Add-Refs,\econtexRoot/\texname}}{
      \typeout{bibliography{\econtexRoot/\texname,system}}
      \bibliography{        \econtexRoot/\texname,system}}
  } % end of picking the one that exists
} % end of testing whether neither exists
}

\ifthenelse{\boolean{Web}}{}{
  \onlyinsubfile{\captionsetup[figure]{list=no}}
  \onlyinsubfile{\captionsetup[table]{list=no}}
  \end{document}	\endinput
}

