% -*- mode: LaTeX; TeX-PDF-mode: t; -*- # Tell emacs file type (for syntax coloring)
% Add the listed directories to the search path
% (allows easy moving of files around later)
% these paths are searched AFTER local config kpsewhich

% *.sty, *.cls
\makeatletter
\def\input@path{{@resources/texlive/texmf-local/tex/latex/}
        ,{@resources/texlive/texmf-local/bibtex/bst/},
        ,{@resources/texlive/texmf-local/bibtex/bib/},
        ,{@local/}
        }
\makeatother
\makeatletter
\def\bibinput@path{{@resources/texlive/texmf-local/tex/latex/}
        ,{@resources/texlive/texmf-local/bibtex/bst/},
        ,{@resources/texlive/texmf-local/bibtex/bib/},
        ,{@local/}
        }
\makeatother
  % allow latex to find custom stuff
% LaTeX path to the root directory of the current project, from the directory in which this file resides
% and path to econtexPaths which defines the rest of the paths like \FigDir
\providecommand{\econtexRoot}{}\renewcommand{\econtexRoot}{.}

\documentclass[\econtexRoot/Letter]{subfiles}
\onlyinsubfile{\externaldocument{\econtexRoot/Letter}} % Get xrefs -- esp to apndx -- from main file; only works if main file has already been compiled

\begin{document}
\notinsubfile{\renewcommand{\econtexRoot}{.}}

%\hypertarget{job-market-paper}{}
%\par\section{Job Market Paper}
%\notinsubfile{\label{sec:job-market-paper}}

Will possesses expertise in building and solving general equilibrium heterogeneous agent macroeconomic models that incorporate realistic micro-level consumption behavior. This expertise could provide significant value to the IMF by enabling precise quantification of fiscal stimulus impacts. For instance, as my co-author in [*Welfare and Spending Effects of Consumption Stimulus Policies*](https://github.com/llorracc/HAFiscal/blob/master/HAFiscal.pdf), Will developed a HANK model with search and matching frictions to calculate multipliers for unemployment insurance extensions, stimulus checks, and tax cuts.

As further evidence of Will's ability, Will has contributed extensively to [HARK](https://docs.econ-ark.org/Documentation/overview/introduction.html), a toolkit for modeling economic decision-making by heterogeneous agents. He has authored a [notebook](https://github.com/econ-ark/HARK/blob/master/examples/ConsNewKeynesianModel/SSJ_example.ipynb) showcasing how HARK integrates with the [Sequence Space Jacobian toolkit](https://github.com/shade-econ/sequence-jacobian) to solve a HANK model. In fact, this work illustrates how deficit-financed government spending generates significantly larger stimulative effects in a HANK framework compared to TANK or RANK models.

Will employed his technical aptitude for solving HANK models during his internship at the Bank of England to build a HANK model with housing as a discrete choice to quantify the housing wealth channel of monetary policy in the UK ([slides with code](https://github.com/wdu9/HANK_Housing_Block/blob/main/HANK_Housing_Slides%20slides.pdf)). These skills are very general purpose; but structural intertemporal modelling is taking over in many fields and the skills Will has developed are eminently transferrable. For example, an immediate application of the work he's done on housing would be the development a macro model that evaluates how monetary policy is transmitted through the housing market, especially since housing models that can capture the distributional consequences of monetary policy are rare.

Overall, Will's expertise in constructing general equilibrium models to evaluate the effectiveness of fiscal policies makes him a strong candidate for the IMF. His advanced computational skills, developed through integrating microeconomic evidence into these models, enable him to analyze the impact of a wide range of fiscal policies. Notably, he is well-equipped to assess the macroeconomic implications of targeted fiscal policies, such as increasing taxes on households in the highest tax bracket, implementing a wealth tax on those holding the largest share of wealth, or stimulus checks to households in a particular income bracket.



\onlyinsubfile{% Allows two (optional) supplements to hard-wired \texname.bib bibfile:
% system.bib is a default bibfile that supplies anything missing elsewhere
% Add-Refs.bib is an override bibfile that supplants anything in \texfile.bib or system.bib
\provideboolean{AddRefsExists}
\provideboolean{systemExists}
\provideboolean{BothExist}
\provideboolean{NeitherExists}
\setboolean{BothExist}{true}
\setboolean{NeitherExists}{true}

\IfFileExists{\econtexRoot/Add-Refs.bib}{
  % then
  \typeout{References in Add-Refs.bib will take precedence over those elsewhere}
  \setboolean{AddRefsExists}{true}
  \setboolean{NeitherExists}{false} % Default is true
}{
  % else
  \setboolean{AddRefsExists}{false} % No added refs exist so defaults will be used
  \setboolean{BothExist}{false}     % Default is that Add-Refs and system.bib both exist
}

% Deal with case where system.bib is found by kpsewhich
\IfFileExists{/usr/local/texlive/texmf-local/bibtex/bib/system.bib}{
  % then
  \typeout{References in system.bib will be used for items not found elsewhere}
  \setboolean{systemExists}{true}
  \setboolean{NeitherExists}{false}
}{
  % else
  \typeout{Found no system database file}
  \setboolean{systemExists}{false}
  \setboolean{BothExist}{false}
}

\ifthenelse{\boolean{showPageHead}}{ %then
  \clearpairofpagestyles % No header for references pages
  }{} % No head has been set to clear

\ifthenelse{\boolean{BothExist}}{
  % then use both
  \typeout{bibliography{\econtexRoot/Add-Refs,\econtexRoot/\texname,system}}
  \bibliography{\econtexRoot/Add-Refs,\econtexRoot/\texname,system}
  % else both do not exist
}{ % maybe neither does?
  \ifthenelse{\boolean{NeitherExists}}{
    \typeout{bibliography{\texname}}
    \bibliography{\texname}}{
    % no -- at least one exists
    \ifthenelse{\boolean{AddRefsExists}}{
      \typeout{bibliography{\econtexRoot/Add-Refs,\econtexRoot/\texname}}
      \bibliography{\econtexRoot/Add-Refs,\econtexRoot/\texname}}{
      \typeout{bibliography{\econtexRoot/\texname,system}}
      \bibliography{        \econtexRoot/\texname,system}}
  } % end of picking the one that exists
} % end of testing whether neither exists
}

\ifthenelse{\boolean{Web}}{}{
  \onlyinsubfile{\captionsetup[figure]{list=no}}
  \onlyinsubfile{\captionsetup[table]{list=no}}
  \end{document}	\endinput
}

