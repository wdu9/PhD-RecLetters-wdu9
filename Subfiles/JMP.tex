% -*- mode: LaTeX; TeX-PDF-mode: t; -*- # Tell emacs file type (for syntax coloring)
\input{./econtexRoot.texinput}
\documentclass[\econtexRoot/Letter]{subfiles}
\onlyinsubfile{\externaldocument{\econtexRoot/Letter}} % Get xrefs -- esp to apndx -- from main file; only works if main file has already been compiled

\begin{document}
\notinsubfile{\renewcommand{\econtexRoot}{.}}

\hypertarget{job-market-paper}{}
%\par\section{Job Market Paper}
\notinsubfile{\label{sec:job-market-paper}}

%[Description of JMP and any other relevant work] \\

Will's job market paper (he will be on the 2024-25 market) is my favorite kind of exercise: Take some well-established microeconomic fact, incorporate it into a HA macro model, and show that it has interesting macro implications that are not present until you match the micro reality. In Will's case, the micro fact is that exogenous spells of unemployment have a ``scarring effect'' on subsequent earnings. The micro literature has made the usual heroic efforts to establish that the measured scars are not due to some form of selection on unobservables - they are, as best one can tell, a causal consequence of an exogenous unemployment shock.  The simplest interpretation is that the spell results in a negative shock to idiosyncratic human capital.  (Other explanations, like a job ladder or match quality, would have the same consequences). Will embeds a micro structural model of unemployment into a general equilibrium setting, matches the facts on scarring, and explores the consequences in a particular dimension: The consequences of a recession for subsequent output. Will demonstrates that, in the presence of scarring, temporary layoffs become a crucial ingredient in driving a swift recovery from a recession. In a counterfactual, Will shows that if the rise in unemployment during the COVID recession was attributed to permanent layoffs, as opposed to temporary layoffs, output would not have made a complete recovery back to its pre-recessionary trend. Furthermore, Will demonstrates that unemployment scarring explains the near permanent rise in income inequality observed following the Great Recession. This motivates an additional role that temporary layoffs can play -- the prevention of a permanent rise in income inequality that typically follows recessions. Will's paper also assess whether a U.S. fiscal consolidation would have been effective at stabilizing the rising debt to GDP ratio during the Great Recession. He demonstrates that a consolidation in the US would have proven ineffective at reducing debt to GDP as scarring erodes future tax revenues thereby raising pressure on the fiscal deficit.

% In Will's model, as in the data, the stimulative policy results in a much lower permanent drop in GDP than the austerity policy, corresponding to the actual outcomes in the US versus Europe.

%To gauge the magnitude of the effects, Will does an exercise to compare the aftermath of the Great Recession in the United States and in the Euro zone, motivated by the fact that in Europe the byword for the fiscal response to the GR was ``austerity'' while in the US there was considerable stimulus (and a litany of criticism that the US should be pursuing austerity not stimulus). In Will's model, as in the data, the stimulative policy results in a much lower permanent drop in GDP than the austerity policy, corresponding to the actual outcomes in the US versus Europe.

%Consistent with the old literature suggesting that there is a unit root in GDP, Will finds that in his HANK model a recession results in a permanent reduction in the level of GDP because the shocks to human capital are permanent. These results connect not only with the old unit root and ``hysteresis'' literatures, but also with the newer ``secular stagnation'' literature. 



% Put here a link to the abstract of the JMP:

%%% Template for any figures
%\begin{figure}[ht!]
%	\centering
%	\includegraphics[width = 0.8\textwidth]{\FigDir/dur_vs_exp_emp.png}
	%\caption{Durable Expenditure and Expected Unemployment Rate} \label{fig:dur_vs_exp_emp}
%\end{figure}

\onlyinsubfile{\input{\LaTeXInputs/bibliography_blend}}

\ifthenelse{\boolean{Web}}{}{
  \onlyinsubfile{\captionsetup[figure]{list=no}}
  \onlyinsubfile{\captionsetup[table]{list=no}}
  \end{document}	\endinput
}

