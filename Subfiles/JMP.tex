% -*- mode: LaTeX; TeX-PDF-mode: t; -*- # Tell emacs file type (for syntax coloring)
\input{./econtexRoot.texinput}
\documentclass[\econtexRoot/Letter]{subfiles}
\onlyinsubfile{\externaldocument{\econtexRoot/Letter}} % Get xrefs -- esp to apndx -- from main file; only works if main file has already been compiled

\begin{document}
\notinsubfile{\renewcommand{\econtexRoot}{.}}

\hypertarget{job-market-paper}{}
%\par\section{Job Market Paper}
\notinsubfile{\label{sec:job-market-paper}}

%[Description of JMP and any other relevant work] \\

Will's job market paper is my favorite kind of exercise: Take some well-established microeconomic fact, incorporate it into a HA macro model, and show that it has interesting macro implications that are not present until you match the micro reality. In Will's case, the micro fact is that exogenous spells of unemployment have a ``scarring effect'' on subsequent earnings. The micro literature has made the usual heroic efforts to establish that the measured scars are not due to some form of selection on unobservables - they are, as best one can tell, a causal consequence of an exogenous unemployment shock.  The simplest interpretation is that the spell results in a negative shock to idiosyncratic human capital.  (Other explanations, like a job ladder or match quality, would have the same consequences). Will embeds a micro structural model of unemployment into a general equilibrium setting, matches the facts on scarring, and explores the consequences in stagnating growth, raising income inequality, and reducing the effectiveness of fiscal austerity. In particular, he finds that unemployment scarring introduces a dimension that can explain the sluggish growth in GDP following the Great Recession and the swift rebound in GDP from the COVID Recession. Furthermore, he demonstrates that scarring provides a rationale for why income inequality rose permanently following the Great Recession but only increased temporarily following the pandemic. He emphasizes that temporary layoffs complement fiscal policy in explaining the differences in recovery between the last two recessions. In the final exercise of his job market paper, he considers a counterfactual where the U.S. pursues fiscal austerity to reduce its debt-to-GDP. He finds that in the presence of scarring, austerity is substantially less effective in reducing debt-to-GDP.

%in a particular dimension: The consequences of a recession for subsequent output. To gauge the magnitude of the effects, Will does an exercise to evaluate the aftermath of the Great Recession in the United States and in the Euro zone, motivated by the fact that in Europe the byword for the fiscal response to the GR was ``austerity'' while in the US there was considerable stimulus (and a litany of criticism that the US should be pursuing austerity not stimulus). In Will's model, as in the data, the stimulative policy results in a much lower permanent drop in GDP than the austerity policy, corresponding to the actual outcomes in the US versus Europe.

%Consistent with the old literature suggesting that there is a unit root in GDP, Will finds that in his HANK model a recession results in a permanent reduction in the level of GDP because the shocks to human capital are permanent. These results connect not only with the old unit root and ``hysteresis'' literatures, but also with the newer ``secular stagnation'' literature. 



% Put here a link to the abstract of the JMP:

%%% Template for any figures
%\begin{figure}[ht!]
%	\centering
%	\includegraphics[width = 0.8\textwidth]{\FigDir/dur_vs_exp_emp.png}
	%\caption{Durable Expenditure and Expected Unemployment Rate} \label{fig:dur_vs_exp_emp}
%\end{figure}

\onlyinsubfile{\input{\LaTeXInputs/bibliography_blend}}

\ifthenelse{\boolean{Web}}{}{
  \onlyinsubfile{\captionsetup[figure]{list=no}}
  \onlyinsubfile{\captionsetup[table]{list=no}}
  \end{document}	\endinput
}

