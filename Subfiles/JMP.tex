% -*- mode: LaTeX; TeX-PDF-mode: t; -*- # Tell emacs file type (for syntax coloring)
\input{@resources/tex-add-search-paths}  % allow latex to find custom stuff
\input{./econtexRoot.texinput}
\documentclass[\econtexRoot/Letter]{subfiles}
\onlyinsubfile{\externaldocument{\econtexRoot/Letter}} % Get xrefs -- esp to apndx -- from main file; only works if main file has already been compiled

\begin{document}
\notinsubfile{\renewcommand{\econtexRoot}{.}}

\hypertarget{job-market-paper}{}
%\par\section{Job Market Paper}
\notinsubfile{\label{sec:job-market-paper}}

%[Description of JMP and any other relevant work] \\

Will's job market paper is my favorite kind of exercise: Take some well-established microeconomic fact, incorporate it into a macro model, and show that it has important macro implications that are not present until you match the micro reality. In Will's case, the micro fact is that exogenous spells of unemployment have a ``scarring effect'' on subsequent earnings. The micro literature has made the usual heroic efforts to establish that the measured scars are not due to some form of selection on unobservables - they are, as best one can tell, a causal consequence of an exogenous unemployment shock.  The simplest interpretation is that the spell results in a negative shock to idiosyncratic human capital.  (Other explanations, like a job ladder or match quality, would have the same consequences).

Although a large microeconomic literature documents these scars and explores its sources, few macroeconomic papers consider its implications for macroeconomic dynamics. The contribution of Will's job market paper is to quantify the macroeconomic impact of these scars on business cycle fluctuations and in the transmission of fiscal policy. % I want to highlight that scarring has not really been incorporated in our macroeconomic frameworks. There is only \href{https://violante.economics.princeton.edu/sites/g/files/toruqf5621/files/documents/AV_Inclusive.pdf}{Alves and Violante 2023} and \href{https://www.bankofcanada.ca/2024/10/staff-working-paper-2024-39/}{Alves and Violante} that explore the implications of scarring for monetary policy. Furthermore, I want to emphasize that I quantify these scars. Making a claim that unemployment scarring induces slow recoveries is not a contribution. In contrast,  quantifying the macroeconomic impact of these scars is a contribution. 

To bring these facts to macroeconomics, Will embeds a micro structural model of unemployment into a general equilibrium setting, matches the facts on scarring, and explores the consequences, which include: (a) a unit root in GDP that arises from the permanence of the scarring; (b) a permanent (or highly persistent) increase in income inequality; and (c) a reduction in the effectiveness of fiscal austerity.
Specifically, he finds that unemployment scarring introduces a dimension that allows his model to capture both failure of GDP to rebound to its former levels following the Great Recession and the swift rebound in GDP from the COVID Recession. % You might briefly cite the whole unit root/secular stagnation literature here

Furthermore, he shows that scarring provides an explanation for why income inequality seems to have been permanently increased by the Great Recession but only increased temporarily during the pandemic recession. In Will's model (and in the micro literature), temporary layoffs do not cause permanent scars to earning capacity, which is why the large fraction of temporary layoffs during the pandemic was not accompanied by a permanent decline in GDP or a permanent rise in income inequality. In particular, Will demonstrates that if the majority of layoffs during the pandemic had been permanent rather than temporary, GDP would have failed to fully recover, stabilizing at a new trend 2 percent below its pre-recession trajectory.\footnote{As a reference for the magnitude of the 2 percent deviation, GDP settled on a new trend 8-10 percent below its pre-recessionary following the Great Recession (e.g. see \href{https://www.federalreserve.gov/econres/notes/feds-notes/why-is-the-u-s-gdp-recovering-faster-than-other-advanced-economies-20240517.html}{this FEDS note comparing the recovery from the pandemic against the recovery from the Great Recession}, figure 10 in Will's job market paper, or figure 8 in \cite{Christiano2015})}

%GDP would have settled on a new trend that is a 2% deviation below the pre-2020 trend and the income Gini index would have permanently increased by 0.2 percentage points.

In the final exercise of his job market paper, he considers a counterfactual where the U.S. pursues fiscal austerity to reduce its debt-to-GDP. He finds that austerity is four times less effective in reducing debt-to-GDP because of unemployment scarring and induces a long lasting rise in income inequality.







%Will has made significant contributions to HARK, an open-source Python toolkit for solving heterogeneous agent models, with a focus on developing methods to compute heterogeneous agent Jacobians. These Jacobians are essential for solving general equilibrium models with rich microeconomic heterogeneity with the Sequence Space Jacobian methodology by \cite{Auclert2023}. Will has created several Jupyter notebooks illustrating how to use HARK to generate these Jacobians and solve general equilibrium HA models. Notably, he developed a notebook for simulating large heterogeneous agent economies with HARK (\href{https://github.com/econ-ark/HARK/blob/master/examples/ConsNewKeynesianModel/Transition_Matrix_Example.ipynb}{Simulation notebook}) and another for computing heterogeneous agent Jacobians (\href{https://github.com/econ-ark/HARK/blob/master/examples/ConsNewKeynesianModel/Jacobian_Example.ipynb}{Jacobian notebook}). He has also integrated his methods with the \href{https://github.com/shade-econ/sequence-jacobian}{Sequence Space Jacobian toolkit} to solve HANK models (\href{https://github.com/econ-ark/HARK/blob/master/examples/ConsNewKeynesianModel/SSJ_example.ipynb}{HANK notebook}) and a Krusell-Smith model without aggregate uncertainty (\href{https://github.com/econ-ark/HARK/blob/master/examples/ConsNewKeynesianModel/KS-HARK-presentation.ipynb}{Krusell-Smith Notebook}). For two consecutive years, Will has been invited to lecture on Sequence Space Jacobian methods in my Ph.D. computational economics course, for which he created a notebook explaining the mathematics and intuition behind the method (\href{https://github.com/econ-ark/HARK/blob/master/examples/ConsNewKeynesianModel/SSJ_explanation.ipynb}{notebook here}). During his tenure as a Ph.D. intern at the Bank of England, Will has leveraged his computational skills on solving general equilibrium models with rich microeconomic features to solve a HANK model with housing as a discrete choice (\href{https://github.com/wdu9/HANK_Housing_Block}{Slides and Code}). Heterogeneous agent models with discrete choice introduce a discrete continuous interaction that makes these models challenging to solve. An overview of the computational tools Will has developed can be found \href{https://www.william-du.com/computational-tools}{here}.




%Will has also made important contributions to HARK, an open source python toolkit for solving heterogeneous agent models. His contributions have largely focused on developing methods that compute heterogeneous agent Jacobians. These Jacobians are an essential object in solving general equilibrium models with rich microeconomic heterogeneity following the Sequence Space Jacobian methodology of \cite{Auclert2023}.  Will has produced several Jupyter notebooks demonstrating how to use the HARK code to produce these Jacobians and then solve general equilibrium models with rich micro heterogeneity. In particular, he has written notebooks on how to utilize his methods to simulate large heterogeneous agent economies with HARK (\href{https://github.com/econ-ark/HARK/blob/master/examples/ConsNewKeynesianModel/Transition_Matrix_Example.ipynb}{Simulation notebook}) and how to compute heterogeneous agent Jacobians (\href{https://github.com/econ-ark/HARK/blob/master/examples/ConsNewKeynesianModel/Jacobian_Example.ipynb}{Jacobian notebook}). In turn, Will has used the methods he constructed in these notebooks to write additional notebooks on how to combine HARK and the \href{https://github.com/shade-econ/sequence-jacobian}{Sequence Space Jacobian toolkit} to solve HANK models (\href{https://github.com/econ-ark/HARK/blob/master/examples/ConsNewKeynesianModel/SSJ_example.ipynb}{HANK notebook}) and a Krusell Smith model without aggregate uncertainty (\href{https://github.com/econ-ark/HARK/blob/master/examples/ConsNewKeynesianModel/KS-HARK-presentation.ipynb}{Krusell Smith Notebook}). For two consecutive years, I have invited Will to give a lecture on Sequence Space Jacobian methods in my Ph.D. computational economics course. For his lecture, Will created a jupyter notebook detailing the mathematics and intuition of the method (\href{https://github.com/econ-ark/HARK/blob/master/examples/ConsNewKeynesianModel/SSJ_explanation.ipynb}{notebook here}). 









%in a particular dimension: The consequences of a recession for subsequent output. To gauge the magnitude of the effects, Will does an exercise to evaluate the aftermath of the Great Recession in the United States and in the Euro zone, motivated by the fact that in Europe the byword for the fiscal response to the GR was ``austerity'' while in the US there was considerable stimulus (and a litany of criticism that the US should be pursuing austerity not stimulus). In Will's model, as in the data, the stimulative policy results in a much lower permanent drop in GDP than the austerity policy, corresponding to the actual outcomes in the US versus Europe.

%Consistent with the old literature suggesting that there is a unit root in GDP, Will finds that in his HANK model a recession results in a permanent reduction in the level of GDP because the shocks to human capital are permanent. These results connect not only with the old unit root and ``hysteresis'' literatures, but also with the newer ``secular stagnation'' literature. 





% Put here a link to the abstract of the JMP:

%%% Template for any figures
%\begin{figure}[ht!]
%	\centering
%	\includegraphics[width = 0.8\textwidth]{\FigDir/dur_vs_exp_emp.png}
	%\caption{Durable Expenditure and Expected Unemployment Rate} \label{fig:dur_vs_exp_emp}
%\end{figure}

\onlyinsubfile{\input{\LaTeXInputs/bibliography_blend}}

\ifthenelse{\boolean{Web}}{}{
  \onlyinsubfile{\captionsetup[figure]{list=no}}
  \onlyinsubfile{\captionsetup[table]{list=no}}
  \end{document}	\endinput
}

