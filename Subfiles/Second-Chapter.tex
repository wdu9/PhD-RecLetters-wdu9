% -*- mode: LaTeX; TeX-PDF-mode: t; -*- # Tell emacs file type (for syntax coloring)
% LaTeX path to the root directory of the current project, from the directory in which this file resides
% and path to econtexPaths which defines the rest of the paths like \FigDir
\providecommand{\econtexRoot}{}\renewcommand{\econtexRoot}{.}

\documentclass[\econtexRoot/Letter]{subfiles}
\onlyinsubfile{\externaldocument{\econtexRoot/Letter}} % Get xrefs from main file; only works if main file has already been compiled

\begin{document}
\notinsubfile{\renewcommand{\econtexRoot}{.}}

% Details on second chapter

An intriguing implication of Will's job market paper is that unemployment risk acts as a much stronger amplifier of business cycles when households recognize that job loss can leave long-term scars on earnings. However, this amplification channel hinges on how accurately households perceive the increased probability of job loss during a recession. In Will's \href{https://acrobat.adobe.com/id/urn:aaid:sc:VA6C2:3adddfd2-1999-477d-b9f6-b8fbffbf4076}{second chapter}, co-authored with Tao Wang, Xincheng Qiu, and Adrian Monninger, Will and his co-authors use data from the \href{https://www.newyorkfed.org/microeconomics/sce#/}{Survey of Consumer Expectations} (SCE) to assess how households' expectations of job finding and job loss evolve over the business cycle. Since the SCE began in 2013,  the only recession it covers is the COVID recession, Will and his co-authors address this limitation by incorporating data from the \href{https://data.sca.isr.umich.edu/}{Michigan Survey of Consumers}, a longstanding survey of household expectations that dates back to the 1970s. This approach enables them to backcast the job finding and job loss expectations time series to 1978. Interestingly, they first document that expectations on the probability of finding a job are highly correlated with the true job finding probability. Secondly, they find that expectations on the probability of losing a job respond sluggishly to the true job separation rate. They then demonstrate that there is substantial heterogeneity across households in how these expectations on unemployment risk evolve. Notably, they find that the bottom 25th percentile of households, in terms of job-finding expectations, exhibits significantly more volatile beliefs about unemployment risk compared to other households. They calibrate a heterogeneous agent model to their new empirical findings on unemployment expectations to evaluate the extent to which precautionary saving has driven fluctuations in aggregate consumption since the late 1970s. In the model, they find that precautionary saving channel is significant and largely driven by changes in job finding expectations as opposed to job separation expectations. This is due to the fact that expectations on losing a job are sticky while job finding expectations are surprisingly very strongly correlated with the true job finding probability. 



\onlyinsubfile{% Allows two (optional) supplements to hard-wired \texname.bib bibfile:
% system.bib is a default bibfile that supplies anything missing elsewhere
% Add-Refs.bib is an override bibfile that supplants anything in \texfile.bib or system.bib
\provideboolean{AddRefsExists}
\provideboolean{systemExists}
\provideboolean{BothExist}
\provideboolean{NeitherExists}
\setboolean{BothExist}{true}
\setboolean{NeitherExists}{true}

\IfFileExists{\econtexRoot/Add-Refs.bib}{
  % then
  \typeout{References in Add-Refs.bib will take precedence over those elsewhere}
  \setboolean{AddRefsExists}{true}
  \setboolean{NeitherExists}{false} % Default is true
}{
  % else
  \setboolean{AddRefsExists}{false} % No added refs exist so defaults will be used
  \setboolean{BothExist}{false}     % Default is that Add-Refs and system.bib both exist
}

% Deal with case where system.bib is found by kpsewhich
\IfFileExists{/usr/local/texlive/texmf-local/bibtex/bib/system.bib}{
  % then
  \typeout{References in system.bib will be used for items not found elsewhere}
  \setboolean{systemExists}{true}
  \setboolean{NeitherExists}{false}
}{
  % else
  \typeout{Found no system database file}
  \setboolean{systemExists}{false}
  \setboolean{BothExist}{false}
}

\ifthenelse{\boolean{showPageHead}}{ %then
  \clearpairofpagestyles % No header for references pages
  }{} % No head has been set to clear

\ifthenelse{\boolean{BothExist}}{
  % then use both
  \typeout{bibliography{\econtexRoot/Add-Refs,\econtexRoot/\texname,system}}
  \bibliography{\econtexRoot/Add-Refs,\econtexRoot/\texname,system}
  % else both do not exist
}{ % maybe neither does?
  \ifthenelse{\boolean{NeitherExists}}{
    \typeout{bibliography{\texname}}
    \bibliography{\texname}}{
    % no -- at least one exists
    \ifthenelse{\boolean{AddRefsExists}}{
      \typeout{bibliography{\econtexRoot/Add-Refs,\econtexRoot/\texname}}
      \bibliography{\econtexRoot/Add-Refs,\econtexRoot/\texname}}{
      \typeout{bibliography{\econtexRoot/\texname,system}}
      \bibliography{        \econtexRoot/\texname,system}}
  } % end of picking the one that exists
} % end of testing whether neither exists
}

\ifthenelse{\boolean{Web}}{}{
  \onlyinsubfile{\captionsetup[figure]{list=no}}
  \onlyinsubfile{\captionsetup[table]{list=no}}
  \end{document}	\endinput
}

