\input{./econtexRoot.texinput}
\documentclass[\econtexRoot/Letter]{subfiles}
\onlyinsubfile{\externaldocument{\econtexRoot/Letter}} % Get xrefs -- esp to apndx -- from main file; only works if main file has already been compiled

\begin{document}

%\hypertarget{introduction}{}\par\section{Introduction}\notinsubfile{\label{sec:intro}}
\setcounter{page}{0}\pagenumbering{arabic}

%[Things student wants CDC to be sure to say in intro - in particular, a clear explanation of why the dissertation internship fits their research agenda, and why the institution would have an interest in having an intern work on whatever you propose to work on].

William's JMP is closely related to the work of a number of economists at the Federal Reserve Board of Governors. In particular, his JMP concerns the macroeconomic consequences of persistent earnings losses following job displacements with regards to the transmission of unemployment risk as a business cycle amplifier and the transmission of various UI policies as macroeconomic stabilizers. He does so by building and solving a HANK model with search and matching friction as well as human capital dynamics to produce realistic persistent earnings losses following job displacement. This is closely related to the job market papers of Bence Bardoczy, Sebastian Graves, and Chris Huckfeldt. Bence's JMP investigates whether spousal insurance would serve as a powerful automatic stabilizer against the effects of countercyclical unemployment risk, while Sebastian's JMP evaluates whether unemployment risk amplifies business cycles. In William's JMP, he shows that UI extensions are a powerful macro stabilization tool due its expectational transmission and demonstrates that the unemployment risk channel is 2-3 times larger when households are subject persistent earnings losses. In Chris' JMP finds that unemployment scarring is concentrated on workers who switch occupations and justifies how this scarring can arise in a structural model. While Chris' JMP studies the causes of unemployment scarring, William's JMP studies the consequences of unemployment scarring instead.\\

Further, William has both strong interest and experience in solving HANK models with Housing as discrete choice. As a PhD intern as the Bank of England, he solved a HANK model with housing as a discrete choice. I have been made aware that various divisions at the Federal Reserve are interested in housing models and therefore William would make a strong fit in any group looking to build such a model as he would be immediately ready to contribute. \\

Finally, William is well versed with the sequence space jacobian approach to solving HANK models. Both his HANK model in his JMP as well as his housing model at the Bank of England were both solved using the sequence space jacobian toolkit. Moreover, Will has integrated the sequence space jacobian methods into the HARK toolkit by programming methods to produce heterogeneous agent Jacobians with the 'Fake News' Algorithm. Since Bence Bardoczy is one of the creators of the sequence space method and toolkit he would mesh well if they were to work together.


\onlyinsubfile{\input{\LaTeXInputs/bibliography_blend}}

\ifthenelse{\boolean{Web}}{}{
  \onlyinsubfile{\captionsetup[figure]{list=no}}
  \onlyinsubfile{\captionsetup[table]{list=no}}
  \end{document} \endinput
}

