% LaTeX path to the root directory of the current project, from the directory in which this file resides
% and path to econtexPaths which defines the rest of the paths like \FigDir
\providecommand{\econtexRoot}{}\renewcommand{\econtexRoot}{.}

\documentclass[\econtexRoot/Letter]{subfiles}
\onlyinsubfile{\externaldocument{\econtexRoot/Letter}} % Get xrefs -- esp to apndx -- from main file; only works if main file has already been compiled

\begin{document}

%\hypertarget{introduction}{}\par\section{Introduction}\notinsubfile{\label{sec:intro}}
\setcounter{page}{0}\pagenumbering{arabic}

%[Things student wants CDC to be sure to say in intro - in particular, a clear explanation of why the dissertation internship fits their research agenda, and why the institution would have an interest in having an intern work on whatever you propose to work on].

His work concerns the macroeconomic consequences of the persistent earnings losses that follow job displacements (\cite{jacobson1993}); he shows that consumers who perceive these losses have a strong precautionary motive, which induces a much stronger `precautionary multiplier' for business cycle fluctuations than arises in the existing literature, which counterfactually assumes that the income losses due to unemployment are purely transitory (or rather, that they are \textit{believed} to be transitory). He does so by building and solving a rich and realistically calibrated HANK model with search and matching frictions (as well as human capital dynamics that reproduce the facts about persistent earnings losses following job displacement).

Will's JMP is closely related to the work of a number of economists at the Fed, in particular Bence Bardoczy, Sebastian Graves, and Christopher Huckfeldt.  \cite{BardoczySpousal} investigates whether spousal insurance would serve as a powerful automatic stabilizer against the effects of countercyclical unemployment risk, while \cite{gravesUrisk} evaluates whether unemployment risk amplifies business cycles. In Will's JMP, he shows that UI extensions are a powerful macro stabilization tool due its expectational transmission and demonstrates that the unemployment risk channel is 2-3 times larger when households are subject persistent earnings losses. In \cite{huckfeldtUscar} finds that unemployment scarring is concentrated on workers who switch occupations and justifies how this scarring can arise in a structural model. While \cite{huckfeldtUscar} studies the causes of unemployment scarring, Will's work studies the consequences of unemployment scarring instead.\\

Further, Will has both strong interest and experience in solving HANK models with housing as discrete choice. As a PhD intern as the Bank of England, he solved a HANK model decomposed the components of the housing wealth channel (\href{https://github.com/wdu9/HANK_Housing_Block/blob/main/HANK_Housing_Slides\%20slides.pdf}{Slides}).  I know from Fed contacts that a number of different sections/groups there are interested in housing models; Will would make a strong fit in any group looking to build such a model as he would be immediately ready to contribute. \\

Finally, Will is well versed with the sequence space Jacobian (SSJ) approach to solving HANK models (\cite{absrSSJ}). Both the HANK model in his JMP as well as the housing model at the Bank of England were solved using the SSJ toolkit. Moreover, Will has integrated the SSJ methods into the \href{https://github.com/econ-ark/HARK}{HARK} toolkit by programming methods to produce heterogeneous agent Jacobians with the 'Fake News' Algorithm (\cite{absrSSJ}). Since Bence Bardoczy is one of the creators of the sequence space method and toolkit, Will would be keen to seek out Bence and connect with him on these topics.


\onlyinsubfile{% Allows two (optional) supplements to hard-wired \texname.bib bibfile:
% system.bib is a default bibfile that supplies anything missing elsewhere
% Add-Refs.bib is an override bibfile that supplants anything in \texfile.bib or system.bib
\provideboolean{AddRefsExists}
\provideboolean{systemExists}
\provideboolean{BothExist}
\provideboolean{NeitherExists}
\setboolean{BothExist}{true}
\setboolean{NeitherExists}{true}

\IfFileExists{\econtexRoot/Add-Refs.bib}{
  % then
  \typeout{References in Add-Refs.bib will take precedence over those elsewhere}
  \setboolean{AddRefsExists}{true}
  \setboolean{NeitherExists}{false} % Default is true
}{
  % else
  \setboolean{AddRefsExists}{false} % No added refs exist so defaults will be used
  \setboolean{BothExist}{false}     % Default is that Add-Refs and system.bib both exist
}

% Deal with case where system.bib is found by kpsewhich
\IfFileExists{/usr/local/texlive/texmf-local/bibtex/bib/system.bib}{
  % then
  \typeout{References in system.bib will be used for items not found elsewhere}
  \setboolean{systemExists}{true}
  \setboolean{NeitherExists}{false}
}{
  % else
  \typeout{Found no system database file}
  \setboolean{systemExists}{false}
  \setboolean{BothExist}{false}
}

\ifthenelse{\boolean{showPageHead}}{ %then
  \clearpairofpagestyles % No header for references pages
  }{} % No head has been set to clear

\ifthenelse{\boolean{BothExist}}{
  % then use both
  \typeout{bibliography{\econtexRoot/Add-Refs,\econtexRoot/\texname,system}}
  \bibliography{\econtexRoot/Add-Refs,\econtexRoot/\texname,system}
  % else both do not exist
}{ % maybe neither does?
  \ifthenelse{\boolean{NeitherExists}}{
    \typeout{bibliography{\texname}}
    \bibliography{\texname}}{
    % no -- at least one exists
    \ifthenelse{\boolean{AddRefsExists}}{
      \typeout{bibliography{\econtexRoot/Add-Refs,\econtexRoot/\texname}}
      \bibliography{\econtexRoot/Add-Refs,\econtexRoot/\texname}}{
      \typeout{bibliography{\econtexRoot/\texname,system}}
      \bibliography{        \econtexRoot/\texname,system}}
  } % end of picking the one that exists
} % end of testing whether neither exists
}

\ifthenelse{\boolean{Web}}{}{
  \onlyinsubfile{\captionsetup[figure]{list=no}}
  \onlyinsubfile{\captionsetup[table]{list=no}}
  \end{document} \endinput
}

